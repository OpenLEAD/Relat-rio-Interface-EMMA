%-----------------------------------------------------------------------
%
%   UFRJ  - Universidade Federal do Rio de Janeiro
%   COPPE - Coordenação dos Programas de Pós-graduação em Engenharia
%   PEE   - Programa de Engenharia Elétrica
%
%
%   Projeto EMMA - 
%
%                                                        19/out/15, Rio
%                                                        Estevão F. Ferrão
%----------------------------------------------------------------------
%
%  Compilar usando PDFLaTeX
%
%----------------------------------------------------------------------
\documentclass[12pt,a4paper]{article}
\usepackage{macros/ROSApackages}
\usepackage[brazilian]{babel}

\input{macros/ROSAsettings}
\input{macros/ROSAmacros}

%\def\PATH{file:c:/Users/Ramon/My Documents/projetos/2015/Projeto EMMA}
\begin{document}
\include{00_Frontpage}

\tableofcontents



\section{Relatório de Usabilidade}

Este relatório tem o objetivo de documentar o processo de pesquisa para o desenvolvimento de software e
interface de usuário do robô EMMA. O estudo irá abordar a funcionalidade necessária para o que o software
em questão atenda aos requisitos funcionais e não funcionais do sistema que controlará o robô. A pesquisa
da usabilidade será abordada sob a perspectiva da interação entre homem-máquina, traçando um paralelo 
entre os usuários do sistema e as tarefas que serão executadas pelo robô.

\section{Objetivos}

O objetivo do projeto EMMA é o de criar uma metodologia para execução de manipulação industrial em ambientes
confinados, reduzindo o risco humano da manutenção e o tempo de parada de máquinas. Essa solução se baseia 
em produtos industriais já existentes para desenvolver uma metodologia que integre os mesmos no ambiente 
confinado. 
O robô EMMA será contituído de um manipulador mecânico Motoman MH12,
uma base mecânica modeluar com três graus de liberdade, um sensor laser scanner FAROXXX e um computador 
para visualizar a interface com usuário. Desta forma o software EMMA irá integrar todos os agentes da operação 
e monitorar as tarefas de calibração, planejamento de trajetória e revestimento realizadas pelo robô.

\section{Pesquisa de usuários}
A pesquisa de usuário identifica todos os atores possíveis no contexto de uso do software 
assim como coleta dados de possíveis usuários para aprender sobre suas características, 
necessidades e preferências. Seu objtivo é estruturar casos de uso e elaborar arquétipos que
sirvam como referência no modelo de arquitetura e nos requisitos do sistema.
Os usuários do projeto EMMA são engenheiros envolvidos no desenvolvimento do projeto assim como 
o time que receberá treinamento na Usina hidrelétrica de Jirau para executar a manutenção de turbinas. 
Em virtude da especificidade de seus usuários e a fim evitar auto referência no desenvovimento do 
sistema, foi feita uma pesquisa com um grupo de 12 engenheiros da área de software, mecânica e
eletrônica para coletar informações e atribuir a um perfil de usuário as referências de construção
do sistema. Através de formulários foram coletadas informações demográficas, comportamentais e 
operacionais para criar uma narrativa que auxilia o entendimento do software criado a partir de 
seus usuários.

\subsection {Persona}


\section{Tarefas do robô}
O sistema EMMA executará três tarefas: calibração, planejamento de trajetória e metalização.

\subsection {Calibração}
Através de um laser scan e sensores espalhados no ambiente do aro câmara é obtida uma nuvem de pontos, 
isto é, uma representação tridimensional do espaço cartesiano, na qual cada distância medida pelos 
sensores a partir de sua origem representa uma coordenada x y z. A identificação de cada conjunto, 
na nuvem de pontos, é importante para que a posição e orientação de cada objeto de interesse seja 
determinada e, assim a transformada do sistema de coordenadas entre cada objeto seja calculada. 
Esses dados possibilitarão ou não um planejamento de trajetória. A etapa de calibração será efetuada 
toda vez que o robô precisar ser posicionado ou reposicionado no ambiente do aro câmera.

\subsubsection {Fluxograma Calibração}

\subsection {Planejamento de Trajetória}
Planejamento de trajetória define como o robô irá executar a aplicação de revestimento nas pás. 
A partir das posições do robô em relação a pá são geradas trajetórias que o efetuador irá percorrer 
para realizar o revestimento em toda a superfície da pá. Cada configuração do manipulador se relaciona 
com uma lista de ângulos das juntas do robô. A aplicação de metalização é dividida em etapas, onde a 
área da pá é particionada.  Para cada uma das etapas são geradas n configurações gerando assim uma matriz 
de ângulo dessas juntas.

\subsubsection {Fluxograma Planejamento de Trajetória}

\subsection {Metalização}


\paragraph{Casos de Uso}








\end{document}
