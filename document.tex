%-----------------------------------------------------------------------
%
%   UFRJ  - Universidade Federal do Rio de Janeiro
%   COPPE - Coordenação dos Programas de Pós-graduação em Engenharia
%   PEE   - Programa de Engenharia Elétrica
%
%
%   Projeto EMMA - 
%
%                                                        19/out/15, Rio
%                                                        Julia Ramos Campana
%----------------------------------------------------------------------
%
%  Compilar usando PDFLaTeX
%
%----------------------------------------------------------------------
\documentclass[12pt,a4paper]{article}
\usepackage{macros/ROSApackages}
\usepackage[brazilian]{babel}

\input{macros/ROSAsettings}
\input{macros/ROSAmacros}

%\def\PATH{file:c:/Users/Ramon/My Documents/projetos/2015/Projeto EMMA}
\begin{document}
\include{00_Frontpage}


\tableofcontents
 


\section{Relatório de Usabilidade}

O Robô EMMA é uma solução para o processo de metalização in situ, isto é, o revestimento de pás no ambiente de turbina, 
diminuindo o tempo de manutenção e, consequentemente, de máquina parada. Esta pesquisa é divida em duas partes; a viabilidade 
técnica e execução. Como parte da viabilidade técnica, a pesquisa de interface de usuários do projeto EMMA propõe diretrizes 
para a elaboração da interface gráfica do sistema de controle do Robô EMMA. Este Relatório aborda os aspectos técnicos que 
dão embasamento para o design de interfaces do sistema e a caracterização de suas funcionalidades, assim como fundamentos 
teóricos que das interações entre homem e máquina.


\section{Objetivo}

O objetivo deste relatório é fundamentar os aspectos técnicos e teóricos que envolvem a interface de usuário do sistema de 
controle do robô EMMA com os seguintes tópicos: modelo de Interação humano-automação, modelagem do sistema, modelos conceitual 
de interface de usuários.

\section{Modelo de Interação}

Automação é um termo amplo, que pode ter muitos significados dependendo do
contexto em que está inserido. Neste estudo, entende-se por automação a realização de tarefas e atividades com a ação de máquinas, sem a interferência direta do humano 
no processo operacional. A interação humano-automação se dá em circunstâncias em que as pessoas podem: (1) especificar 
para a automação a tarefa (através de computador de algum tipo), objetivos e restrições e trade-offs ou trocas entre os 
objetivos e restrições; (2) controlar a automação para iniciar, parar ou modificar a tarefa automática de execução; e 
(3) receber a partir da informação de automação, energia, objetos físicos, ou substâncias.

Desta forma os aspectos que permeiam a interação humano-automatização envolvem tanto o homem quanto a máquina numa alimentação mútua, 
onde a idéia central é que o sistema fornece ao homem determinada informação, que a processa e realimenta o sistema com tal informação 
transformada, ou seja de acordo com as necessidades de uso do sistema. 
Um modelo que define a interação humano-automação de forma mais detalhada é chamado de OODA LOOP (Observation, Orientation, Decision, Action) 
\cite{sheridan2005human} que leva em
consideração não só os processos de interação, mas também elementos (subjetivos e concretos) que interferem em cada uma das etapas. 

O OODA LOOP (fig.\ref{fig:ooda}) é um esquema com uma série de ações
cognitivas e motorasrequeridas de um operador/usuário para tomar decisões e executá-las durante a interação. Este esquema leva em consideração aspectos referentes ao usuário, como experiências prévias, aspectos culturais, informação 
dada ao usuário e de canais de atenção. A análise e consideração de todas essas variáveis torna-se essencial para que o usuário possa entender o 
que se passa durante a ação de um de um sistema automatizado e assim agir corretamente diante de cada situação. Da mesma forma, sinaliza a 
importância de estabelecer um perfil usuário baseado naqueles que usarão de forma direta ou indireta e também influenciarão em aspectos do projeto.

\begin{figure}[h!]
\begin{center}
\fbox {\includegraphics[width=\columnwidth]{figs/ooda_loop.png}}
  \caption{Esquema OODA Loop}
  \label{fig:ooda}
\end{center}
\end{figure} 
 
O esquema OODA LOOP também é aplicado para o funcionamento dos sistemas em si, levando em consideração as variáveis, cenários pré-determinados e 
leituras dos sensores para tomar as suas decisões. Durante a comunicação entre os usuários e o sistema, os modelos das duas partes estão diretamente 
interligados, uma vez que a falha de interpretação por parte dos sensores ou um feedback impreciso interfere diretamente nas ações do operador.
Ao observar o modelo acima (fig.\ref{fig:ooda}), pode-se perceber que fatores humanos subjetivos e culturais são de suma importância para o bom desempenho 
das tarefas envolvendo elementos automatizados - uma vez que a interpretação dos estados do sistema e do funcionamento do mesmo está em função das 
experiências prévias do usuário, assim como antecedentes culturais relacionados a automatização. Diversos autores 
\cite{Parazuraman2008,degani2003analysis,sheridan2005human} afirmam que um dos
principais fatores que influenciam a interação Humano-Automatização são as experiências prévias dos usuários com este tipo de sistema. Na prática o modelo nos fornece elementos cognitivos (observação, orientação, decisão e ação) para compreender as escolhas do usuário a partir da leitura que o ambiente e seus dispositivos. Ademais, 
nos fornece aspectos relevantes a serem considerados tanto no processo de envolvimento de usuários no projeto e consequentemente no design centrado 
no usuário.


\section{Modelagem do Sistema}

A meta principal do sistema de controle robótico EMMA é a de realizar metalização in situ, isto é, o revestimento de pás no ambiente de turbina, 
diminuindo o tempo de manutenção e, consequentemente, de parada de máquinas.


\subsection{Requistos do Sistema}
Um manipulador robótico instalado no ambiente do circuito hidráulico onde o
revestimento deve acontecer, incluindo os sensores para medições necessárias e as ferramentas para executar a tarefa.

\subsection{Problema}
\begin {enumerate}
  \item Determinar ás áreas a serem revestidas.
  \item Instalar o manipulador na posição correta para realizar o revestimento.
  \item Calcular e validar a posição instalada do manipulador.
  \item Calcular como o manipulador deve se mover para realizar a o
  revestimento.
  \item Executar o revestimento.
\end{enumerate}

\subsection{Funções do Sistema}
\begin {enumerate}
  \item \textbf{Análise das Pás:} A função da tarefa de análise é analisar as
  informações dos sensores para determinar os desgastes mecânicos e as regiões que precisam de revestimento.
  \item \textbf{ValidaçÃo:} Auxiliar o usuário a posicionar o robô de forma
  ótima, ou seja, a posição na qual ele possui a maior cobertura para realizar o revestimento das áreas determinadas na análise revestimento.
  \item \textbf {Calibração:} A função da tarefa de calibração é calcular e
  validar a posição relativa entre o manipulador e a pá. Ou seja, se usuário montou o manipulador na posição correta estimada pelo alinhamento.
  \item \textbf {Trajetória:} A tarefa de trajetória tem como função calcular o
  movimento que o manipulador precisa executar para realizar o revestimento.
  \item \textbf {Execução:} Execução dos movimentos estimados na trajetória para
  executar o revestimento.
\end{enumerate}

\subsection{Componentes do Sistema}
A solução EMMA é formada de partes distintas:
\begin {enumerate}
	\item um manipulador robótico
	\item um scanner laser 3D
	\item uma base mecânica com trilho
	\item uma pistola de coating
	\item medidor de distância a laser
	\item software de interface e controle.
\end {enumerate} 

\subsection {Ambiente do Sistema}
O ambiente onde a operação de metalização ocorre é um ambiente confinado, o circuito hidráulico onde as turbinas, e consequentemente as pás, se encontram. 
Este ambiente impõe restrições que forma que determinantes na escolha de componentes e planejamento da solução.

\subsection {Utilização do Sistema}
\begin {enumerate}
  \item O usuário acessa o circuito hidráulico realiza medições com o laser scanner 3D e faz fotos das pás.
  \item Essas informações são carregas para o software na etapa da análise.
  \item O usuário analise estes dados e determina as áreas a serem revestidas.
  \item Após analisado, o software calcula o alinhamento ótimo do manipulador
para acessar essas áreas.
  \item O usuário instala o trilho e a base, posicionando o manipulador de
  acordo com o resultado do alinhamento.
  \item O usuário utiliza novamente o laser scanner 3D medindo as posições do
manipulador e da pá.
  \item O usuário carrega estes dados ao software que calcula a posição relativa
  entre o manipulador e a pá (Calibração).
  \item Está calibração é validada através de movimentos do manipulador com
  malha de controle fechada usando o medidor de distância laser.
  \item Uma vez calibrado o software calcula as trajetórias que serão feitas
  para realizar o revestimento das áreas desejadas.
  \item O usuário monitora se este movimento e as forças e velocidade geradas no
  mesmo estão de acordo funcionamento esperado.
  \item O usuário manda executar estas trajetórias, realizando assim o
  revestimento. 
\end{enumerate}



\subsection {Diagramas de Caracterização do Sistema}
A caracterização do sistema acontece para que se determine com clareza as
funcionalidades de cada tarefa do sistema. O sistema alvo situa-se numa posição serial e recebe entradas de 
uma sistema que lhe é anterior - o sistema alimentador - e, por sua vez, produz saídas para um sistema que 
lhe é posterior - o sistema ulterior. As entradas são processadas pelo processo característico do sistema alvo.

\begin{enumerate}
	
	\item \textbf {Sistema Alimentador:}Sistema que fornece as entradas para o
	sistema alvo.
	\item \textbf {Entradas:} Elementos que serão processados pelo sistema
	(matéria-prima ou informações).
	\item \textbf {Sistema-Alvo:} Sistema homem-tarefa-máquina analisado.
	\item \textbf {Saídas:} Resultados do processo realizado pelo sistema-alvo.
	\item \textbf {Sistema Ulterior:} Sistema que recebe as saídas do sistema-alvo.
	\item \textbf {Requisitos:} O que o sistema deve ter para funcionar.
	\item \textbf {Metas:} Determinam para que serve o sistema; quais os
	requisitos, quais atributos o sistema de ter ou pertencer para que sua meta seja atingida.
	\item \textbf {Restrições:} define coações fixas que que estão no ambiente do
	sistema e criam obstáculos para a implementação dos requisitos.

\end{enumerate}

\subsubsection {Análise}

\begin{figure}[H]
\begin{center}
  \includegraphics[width=0.95\textwidth]{figs/caracterizacao_analise.jpg}
  \caption{Caracterização da função de análise.}
  \label{fig:caracterizacao_analise}
\end{center}
\end{figure} 


\subsubsection {Calibração}

\begin{figure}[H]
\begin{center}
  \includegraphics[width=0.95\textwidth]{figs/caracterizacao_calibracao.jpg}
  \caption{Caracterização da função de calibração.}
  \label{fig:caracterizacao_calibracao}
\end{center}
\end{figure} 

\subsubsection {Validação}

\begin{figure}[H]
\begin{center}
  \includegraphics[width=0.95\textwidth]{figs/caracterizacao_validacao.jpg}
  \caption{Caracterização da função de validação.}
  \label{fig:caracterizacao_validacao}
\end{center}
\end{figure} 

\subsubsection {Trajetória}

\begin{figure}[H]
\begin{center}
  \includegraphics[width=0.95\textwidth]{figs/caracterizacao_trajetoria.jpg}
  \caption{Caracterização da função de trajetória.}
  \label{fig:caracterizacao_trajetoria}
\end{center}
\end{figure} 

\subsubsection {Execução}

\begin{figure}[H] 
\begin{center} 
  \includegraphics[width=0.95\textwidth]{figs/caracterizacao_execucao.jpg}
  \caption{Caracterização da função de execução.}
  \label{fig:caracterizacao_execucao}
\end{center}
\end{figure} 


\section {Modelo Conceitual de Interface Gráfica}

\subsection {Esqueleto do Sistema}
O planejamento da interface gráfica tem início em um desenho básico, chamado
esqueleto, um esqueleto que demonstra de forma direta a arquitetura de como a
interface do software EMMA vai ser visualizada.

\begin{figure}[H]
\begin{center}
  \fbox{\includegraphics[width=0.95\textwidth]{figs/wireframe.png}}
  \caption{Desenho do esqueleto do sistema.}
  \label{fig:esqueleto_sistema}
\end{center}
\end{figure} 

A idéia deste esqueleto é de representar no espaço definido para interface
gráfica dentro da lógica de caracterizão do sistema. Desta forma relaciona
diretamente entradas e saídas de informação, com as metas almejadas, e
permitindo que o usuário visualize a informaçào necessária para dar
prosseguimento a operação.


\subsection {Desenhos de Interface}

\subsubsection {Análise}

\begin{figure}[H]
\begin{center}
  \includegraphics[width=.95\columnwidth]{figs/Analise.jpg}
  \caption{Modelo conceitual de interface de usuário para função de análise.}
  \label{fig:Interface_analise}
\end{center}
\end{figure} 

A (fig.\ref{fig:Interface_analise}) mostra a interface gráfica para usuário da
tarefa de análise, que tem como objetivo avaliar o desgaste mecânico de pás
fornecendo um relatório para seu revestimento posterior. Assim a informação do laser scanner 3D
fornece medidas do ambiente sob a forma de nuvem de pontos para que se tenha um
detalhamento da pá analisada. Ademais, também é possível adicionar informação
externa sob a forma de fotografia para dar mais detalhes e precisão a
determinado desgaste mecânico.
Assim, através do uso de ferramentas disponibilizadas na área destinada a meta
do software, o usuário pode selecionar e visualizar áreas onde existem
danos, fazendo anotações e descrevendo ações necessárias para a manutenção de
determinada pá.

\subsubsection {Alinhamento}

\begin{figure}[H]
\begin{center}
  \includegraphics[width=.95\columnwidth]{figs/Validacao.jpg}
  \caption{Modelo conceitual de interface de usuário função de validação.}
  \label{fig:Interface_validacao}
\end{center}
\end{figure} 

A (fig.\ref{fig:Interface_validacao}) mostra a interface gráfica para usuário
da tarefa de alinhamento. A tarefa de alinhamento permite que se estime novas
posições do manipulardor em relação a pá, permitindo que o robô
seja posicionado de forma ótima. Ao reposicionar o robô na interface é possível
recalcular sua posição, descobrindo onde o manipulador terá a maior cobertura
para realizar o revestimento das áreas determinadas pela análise.
Na interface se encontram ferramentas que permitem a movimnetação do robô em
eixo determinado para o cálculo. Assim, o susário poderá reposicionar o robô de
acordo, obtendo as informações necessárias a calibracão e consequentemente o
cálculo de trajetórias.


\subsubsection {Calibração}

\begin{figure}[H]
\begin{center}
  \includegraphics[width=.95\columnwidth]{figs/Calibracao.jpg}
  \caption{Modelo conceitual de interface de usuário para função de calibração.}
  \label{fig:Interface_calibracao}
\end{center}
\end{figure} 

A (fig.\ref{fig:Interface_calibracao}) mostra a interface gráfica para usuário
da tarefa de calibração, que tem como objetivo mostrar a posição relativa do
manipulador em relação a pá. Caso o erro seja maior que o limite determinado
este valor e mostrado em vermelho, sinalizando novas medições o laser scanner
para determinar posiçào que satisfaça o sistema e permita o calculo de
trajetória.


\subsubsection {Trajetória}

\begin{figure}[H]
\begin{center}
  \includegraphics[width=\columnwidth]{figs/Trajetoria.jpg}
  \caption{Modelo conceitual de interface de usuário função de trajetória.}
  \label{fig:Interface_trajetoria}
\end{center}
\end{figure} 

A (fig.\ref{fig:Interface_trajetoria}) mostra a interface gráfica para usuário
da tarefa de trajetória. O objetivo desta tarefa é a de determinar a região onde
o manipulador pode atuar no revestimento. Desta forma o sistema recebe
informações do cálculo de erro de transformada, ou seja, a posição relativa
entre pá e manipulador para a partir destas distâncias calcular a área de
atuação do manipulador que efetuará o revestimento.

\bibliographystyle{unsrt}
\bibliography{main} 



\end{document}
