%-----------------------------------------------------------------------
%
%   UFRJ  - Universidade Federal do Rio de Janeiro
%   COPPE - Coordenação dos Programas de Pós-graduação em Engenharia
%   PEE   - Programa de Engenharia Elétrica
%
%
%   Projeto EMMA - 
%
%                                                        19/out/15, Rio
%                                                        Estevão F. Ferrão
%----------------------------------------------------------------------
%
%  Compilar usando PDFLaTeX
%
%----------------------------------------------------------------------
\documentclass[12pt,a4paper]{article}
\usepackage{macros/ROSApackages}
\usepackage[brazilian]{babel}

\input{macros/ROSAsettings}
\input{macros/ROSAmacros}

%\def\PATH{file:c:/Users/Ramon/My Documents/projetos/2015/Projeto EMMA}
\begin{document}
\include{00_Frontpage}

\tableofcontents



\section{Objetivo}

Este relatório tem o objetivo de documentar o processo de pesquisa na áerea de
Interfaces de Usuário do projeto EMMA. A primeira fase do projeto entre Março de
2015 e Abril de 2016 prevê um estudo de viabilidade técnica para a construção de
um robô que realizará manutenção em turbinas de bulbo nas Usinas de Jirau e
Santo Antônio em Porto Velho (RO). O robô EMMA será dotado de um sistema
operacional com uma interface gráfica que controlerá suas atividades.

\section{Processo de Design}

A idéia de construir uma interface gráfica para usuários 
técnicos e diretamente ligados ao desenvolvimento da mesma estabelece um desafio
no que se refere ao design centrado no usuário (user centered design). A
intenção é que a referência própria exista até certo ponto, mas em um processo
de design centrado no usuário a objetiva-se que determinados arquétipos de
usuários, traduzidos em personas, reúnam características que sejam capazes de
justificar os aspectos utilizados no desenvolvimento. Partindo desse princípio o
trabalho desenvolvido no projeto EMMA será dividido em três fases: Descoberta, Design e Desenvolvimento.


\subsection{Descoberta}

A fase de descoberta consiste em entender exatamente o que esta sendo construído
e seus objetivos enquanto produtos. Em um primeiro momento 

\subsection{Design}

\subsection{Desenvolvimento}


\end{document}
