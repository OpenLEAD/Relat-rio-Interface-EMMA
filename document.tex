%-----------------------------------------------------------------------
%
%   UFRJ  - Universidade Federal do Rio de Janeiro
%   COPPE - Coordenação dos Programas de Pós-graduação em Engenharia
%   PEE   - Programa de Engenharia Elétrica
%
%
%   Projeto EMMA - 
%
%                                                        19/out/15, Rio
%                                                        Estevão F. Ferrão
%----------------------------------------------------------------------
%
%  Compilar usando PDFLaTeX
%
%----------------------------------------------------------------------
\documentclass[12pt,a4paper]{article}
\usepackage{macros/ROSApackages}
\usepackage[brazilian]{babel}

\input{macros/ROSAsettings}
\input{macros/ROSAmacros}

%\def\PATH{file:c:/Users/Ramon/My Documents/projetos/2015/Projeto EMMA}
\begin{document}
\include{00_Frontpage}

\tableofcontents



\section{Relatório de Usabilidade}

O Robô EMMA é uma solução para o processo de metalização in situ, isto é, o revestimento de pás no ambiente de turbina, 
diminuindo o tempo de manutenção e, consequentemente, de máquina parada. Esta pesquisa é divida em duas partes; a viabilidade 
técnica e execução. Como parte da viabilidade técnica, a pesquisa de interface de usuários do projeto EMMA propõe diretrizes 
para a elaboração da interface gráfica do sistema de controle do Robô EMMA. Este Relatório aborda os aspectos técnicos que 
dão embasamento para o design de interfaces do sistema e a caracterização de suas funcionalidades, assim como fundamentos 
teóricos que das interações entre homem e máquina.


\section{Objetivo}

O objetivo deste relatório é fundamentar os aspectos técnicos e teóricos que envolvem a interface de usuário do sistema de 
controle do robô EMMA com os seguintes tópicos: modelo de Interação humano-automação, modelagem do sistema, modelos conceitual 
de interface de usuários.

\section{Modelo de Interação}

Automação é um termo amplo, que pode ter muitos significados dependendo do
contexto em que está inserido. Neste estudo, entende-se por automação a realização de tarefas e atividades com a ação de máquinas, sem a interferência direta do humano 
no processo operacional. A interação humano-automação se dá em circunstâncias em que as pessoas podem: (1) especificar 
para a automação a tarefa (através de computador de algum tipo), objetivos e restrições e trade-offs ou trocas entre os 
objetivos e restrições; (2) controlar a automação para iniciar, parar ou modificar a tarefa automática de execução; e 
(3) receber a partir da informação de automação, energia, objetos físicos, ou substâncias.

Desta forma os aspectos que permeiam a interação humano-automatização envolvem tanto o homem quanto a máquina numa alimentação mútua, 
onde a idéia central é que o sistema fornece ao homem determinada informação, que a processa e realimenta o sistema com tal informação 
transformada, ou seja de acordo com as necessidades de uso do sistema. 
Um modelo que define a interação humano-automação de forma mais detalhada é chamado de OODA LOOP (Observation, Orientation, Decision, Action) 
(Thomas, 2001 apud Gikkas, 2013) que leva em consideração não só os processos de interação, mas também elementos (subjetivos e concretos) que 
interferem em cada uma das etapas. 

O OODA LOOP (fig. 1) é um esquema com uma série de ações cognitivas e motorasrequeridas de um operador/usuário para tomar decisões e executá-las 
durante a interação. Este esquema leva em consideração aspectos referentes ao usuário, como experiências prévias, aspectos culturais, informação 
dada ao usuário e de canais de atenção. A análise e consideração de todas essas variáveis torna-se essencial para que o usuário possa entender o 
que se passa durante a ação de um de um sistema automatizado e assim agir corretamente diante de cada situação. Da mesma forma, sinaliza a 
importância de estabelecer um perfil usuário baseado naqueles que usarão de forma direta ou indireta e também influenciarão em aspectos do projeto.

\begin{figure}[htp]
\begin{center}
  \includegraphics[width=figureWidth]{oodaLoop.jpg}
  \caption[labelInTOC]{figureCaption}
  \label{figureLabel}
\end{center}
\end{figure}
 

O esquema OODA LOOP também é aplicado para o funcionamento dos sistemas em si, levando em consideração as variáveis, cenários pré-determinados e 
leituras dos sensores para tomar as suas decisões. Durante a comunicação entre os usuários e o sistema, os modelos das duas partes estão diretamente 
interligados, uma vez que a falha de interpretação por parte dos sensores ou um feedback impreciso interfere diretamente nas ações do operador.
Ao observar o modelo acima (fig. 2), pode-se perceber que fatores humanos subjetivos e culturais são de suma importância para o bom desempenho 
das tarefas envolvendo elementos automatizados - uma vez que a interpretação dos estados do sistema e do funcionamento do mesmo está em função das 
experiências prévias do usuário, assim como antecedentes culturais relacionados à automatização. Diversos autores 
(Parasuraman & Sinderman, 2011; DEGANI, 2003; Thomas apud Gikkas, 2013) afirmam que um dos principais fatores que influenciam a 
interação Humano-Automatização são as experiências prévias dos usuários com este tipo de sistema. Na prática o modelo nos fornece elementos cognitivos 
(observação, orientação, decisão e ação) para compreender as escolhas do usuário a partir da leitura que o ambiente e seus dispositivos. Ademais, 
nos fornece aspectos relevantes a serem considerados tanto no processo de envolvimento de usuários no projeto e consequentemente no design centrado 
no usuário.


\section{Modelagem do Sistema}

A meta principal do sistema de controle robótico EMMA é a de realizar metalização in situ, isto é, o revestimento de pás no ambiente de turbina, 
diminuindo o tempo de manutenção e, consequentemente, de parada de máquinas.


\subsection{Requistos do Sistema}
É ter um manipulador robótica instalado no ambiente do circuito hidráulico onde o revestimento deve acontecer, incluindo os sensores para medições 
necessárias e as ferramentas para executar a tarefa.

\subsection{Problema}
\begin {enumerate}
  \item Determinar ás áreas a serem revestidas.
  \item Instalar o manipulador na posição correta para realizar o revestimento.
  \item Calcular e validar a posição instalada do manipulador.
  \item Calcular como o manipulador deve se mover para realizar a o
  revestimento.
  \item Executar o revestimento.
\end{enumerate}





\end{document}
